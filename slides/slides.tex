\documentclass{beamer}
\usetheme{Boadilla}
\usepackage[utf8]{inputenc}

\usepackage{amsfonts,amssymb, amsthm, amsmath}
\usepackage{enumerate}
\usepackage[utf8]{inputenc}
\usepackage[english,russian]{babel}
\usepackage{graphicx}
\usepackage{multicol}
\usepackage{caption}
\usepackage{ textcomp }
\usepackage{url}
\usepackage{ bbold }
\usepackage{ mathrsfs }
\usepackage{ dsfont }


\title[Дипломная работа]{Исследование влияния инициализации слоев бэкбона в задаче обнаружения аномалий}
\author[Щекалёв Михаил]{Щекалев Михаил, 611 группа \\ Научный руководитель: к.ф.-м.н., с.н.с. Мазуренко Иван Леонидович}
\institute[]{МГУ имени М.В. Ломоносова \\ механико-математический факультет \\ кафедра математической теории интеллектуальных систем}

\begin{document}
	
\maketitle

\begin{frame}{Задача обнаружения аномалий}
	Пусть $X \subset \mathbb{R}^N$ - множество изображений нормальных объектов.
	
	\
	
	Задача обнаружения аномалий: имея только $X$ построить $p_X \ : \mathbb{R}^N \ \rightarrow \ [0, +\infty)$ - коеффициент аномальности.
	
	\
	
	Задача локализации аномалий: имея только $X$ построить для каждого $i$-го измерения (пикселя) $p_{i, X} \ : \mathbb{R} \ \rightarrow \ [0, +\infty)$, $i = 1,..., N$
\end{frame}

\begin{frame}{Методы сравнения эмбеддингов}
	Класс данных методов заключается в построении отображения (бэкбона) $F \ : \ \mathbb{R}^N \ \rightarrow \ \mathbb{R}^K$, который для каждого $x \in X$ строит вектор (эмбеддинг) $a_x \in \mathbb{R}^K$. По построенным векторам собирается множество $A(X) = \{a_x\}_{x \in X}$.
	
	Далее по определённой логике, индивидуальной для каждого метода, строится коеффициент аномальности $p_{F, X} \ : \mathbb{R}^N \ \rightarrow \ [0, +\infty)$.
	
	\
	
	Исследуемые в работе методы: SPADE, PaDiM. Требуется для ряда упомянутых методов исследовать влияние выбора бэкбона на целевые метрики (о них ниже).
\end{frame}

\begin{frame}{Цели работы}
	Исследование влияния выбора архитектуры глубоких нейронных сетей и логики извлечения признаков, составляющих бэкбон, на качество и производительность наиболее релевантных и сбалансированных с точки зрения сложности и качества методов решения задачи обнаружения (и локализации) аномалий в области компьютерного зрения.
\end{frame}

\begin{frame}{Этапы работы}
	\begin{itemize}
		\item Изучение методов решения задачи обнаружения;
		\item Изучение архитектур нейронных сетей;
		\item Программная реализация методов с возможностью использования изученных архитектур и варьированием гиперпараметров бэкбона;
		\item Проведение экспериментов, исследование влияния выбора на значения целевых метрик;
		\item Интерпретация результатов, подведение итогов.
	\end{itemize}
\end{frame}

\begin{frame}{Детали эксперимнтов}
	\begin{itemize}
		\item ROC-AUC -- соотношение между долей верно классифицированных как аномальные объектов от общего количества и долей ошибочно классифицированных как аномальные объектов от общего количества объектов.
		
		\item pROC-AUC -- то же самое, только для пикселей вместо объектов.
		
		\item Датасет для оценки качества -- конвенционально признанный MVTec AD.
	\end{itemize}
\end{frame}

\begin{frame}{Результаты работы}
	Для SOTA методов решения задачи обнаружения (и локализации) аномалий в качестве бэкбона были протестированы более 70 различных вариантов наиболее релевантных и современных в области компьютерного зрения свёрточных нейронных сетей с различными гиперпараметрами. Для каждой из архитектур выбраны гиперпараметры, показывающие наилучший результат в терминах качества и быстродействия.
\end{frame}
	
\end{document}